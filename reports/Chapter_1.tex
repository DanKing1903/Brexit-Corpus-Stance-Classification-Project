\documentclass[Dissertation.tex]{subfiles}
\begin{document}
\chapter{Introduction}
The political landscape of the western world in recent years has changed dramatically. With many nations suffering economic crises and witnessing an influx of refugees fleeing violence and war in the middle east, huge swathes of voters across the western hemisphere are rejecting the prevailing centrist dogma in favour of populist, nationalist, xenophobic right wing movements. In continental Europe nationalist parties now have a significant presence in the governments of Hungary, Poland, Austria, Italy and Germanys 

 %\cite{krisztinaHungaryStrongmanViktor2018}, %\cite{marcinPolandEuroscepticsWin2015}, %\cite{gavinItalyConteSworn2018},
%\cite{kirstiAustriaKurzStrikes2017}, %\cite{michelleGermanyJubilantFarright2017}. % Source required	
In America traditional republican candidates lost the primaries to the demagoguery of Donald Trump, while establishment candidate Hillary Clinton failed to energise the centre-left vote% \cite{ericTrumpWinsElectoral2016}. 

There are a great number of socio-political and economic factors that can explain the expansion of the far right in Europe and America, but modern technology has almost certainly catalysed its rise to prominence. In particular, this is characterised by two related but opposing phenomena: the democratisation of journalism and the proliferation of hyper-polarized echo-chambers. The hegemony of corporate giants and state broadcasters has been disrupted by the explosion of social media and micro-blogging:  world events are now documented in real time on Twitter and Facebook live streams, and any person with a computer can now publish a blog for the world to read. It is easier than ever to access and consume news from countless different sources, yet Gaughan \cite{gaughanIlliberalDemocracyToxic2017} argues that a toxic mix of fake news, hyperpolarization and electoral manipulation has led to partisan divides being greater than ever. Political opinions are polarized and hardened by ranking algorithms that prioritise sensationalism and emotion over veracity and integrity, while internet hyper connectivity provides radical, fringe viewpoints a platform around which to build communities safe from critical debate. 

Following a a long and bitterly fought political campaign spearheaded by the United Kingdom Independence Party (UKIP), in 2016 the world looked on in shock as Britain voted to leave the EU by national referendum. Drawing many parallels to the election victory of Donald Trump, the Brexit referendum was marked by misinformation, vehement partisanship and an outcome that polls failed to predict.

In such politically turbulent times there is significant interest in applying machine learning and natural language processing (NLP) techniques to political social media content, such as opinion mining and sentiment analysis. This project considers the related task of stance detection, which in the simplest case involves determining if a text in agreement or disagreement with a given proposition. As one of the most significant and polarizing political events in a generation, the Brexit debate is an ideal topic for investigating stance detection.

\section{Aims and Objectives}
This project aimed to investigate stance detection using the Brexit Blog Corpus, a publicly available dataset created by Simaki et al \cite{simakiAnnotatingSpeakerStance2017}. The data set is comprised of texts extracted from public blog posts written in 2016 concerning the upcoming British referendum on European Union membership. Neural approaches were investigated and compared to kernel based methods, and the relationship between multi-class, multi-label and multi-task learning problems was explored. Each method was benchmarked using standard metrics and its performance was improved using hyper-parameter tuning techniques.

\section{Report Structure}
To Be Completed

%tell them what you are going to tell them
\end{document}