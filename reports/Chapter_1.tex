\documentclass[Dissertation.tex]{subfiles}
\begin{document}
\chapter{Introduction}
Opinions are like arseholes — everybody's got one. The political landscape of the modern western world is perhaps best characterised by two related but opposing phenomena: the democratisation of journalism and the proliferation of hyper-polarized echo-chambers. The hegemony of media giants and state broadcasters has been disrupted by the explosion of social media and micro-blogging:  world events are now documented in real time on Twitter and Facebook live streams by the persons experiencing them first hand, while any person with a computer can now publish a blog for the world to read. It is easier than ever to access and consume news from countless different sources, yet we do not observe the wider public diversifying their media consumption and gravitating to centrist viewpoints. Instead we observe the exact opposite - the new media paradigms have created societies that are more divided than ever. Political opinions are polarized and hardened by ranking algorithms that prioritise sensationalism and emotion over veracity and integrity, while hyperconnected social networks have provided every fringe group from the Flat-Earth movement to 9/11 conspiracy theorists with a platform upon which to build communities safe from critical discourse and challenging viewpoints.

Across the first world we see these phenomena manifesting in the rise of anti-establishment politics and the collapse of traditional voter bases. In continental Europe nationalist parties now have a significant presence in the governments of Hungary, Poland, Austria, Italy and Germany. In America traditional republican candidates lost the primaries to the demagoguery of Donald Trump, while establishment candidate Hillary Clinton failed to energise the centre-left vote. 

\section{}
\end{document}