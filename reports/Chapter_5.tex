\documentclass[Dissertation.tex]{subfiles}
\begin{document}
\chapter{Results and Analysis}
\section{Introduction}
In this chapter the performance of the models developed is evaluated and compared. The chapter begins with a description of the hyper-parameter tuning process employed for the final FastText models, before the results for each model are presented. Finally, the chapter concludes with an quantitative analysis of model performance and qualitative review of predictions using the anchoring method described in section \ref{sec:anchors}.

\section{Hyper-Parameter Tuning} \label{sec:hypParams}
When developing machine learning models we learn parameters (weights) which are optimized by the learning algorithm. However, there are also hyper-parameters that control the behaviour of the learning algorithm. An important part of machine learning therefore is optimizing hyper-parameters. This is usually accomplished with cross validation. In this project a 'greedy' strategy is employed - once a parameter is tuned, this parameter value is fixed when subsequently tuning other parameters, and no grid search is performed.

\subsection{Optimizer}
TBC

\subsection{Embedding Length}
For the FastText classifier described in section \ref{sec:fastTextMethod} several lengths were tested for the embedding vector. Increasing the embedding length provides a larger parameter space to learn, which should provide greater performance, however this comes at a trade off with run time. In table \ref{tab:embedTuning} are the results of 5 fold cross validation for different embedding lengths. Initially each fold is trained for 100 epochs, though this hyper-parameter will be tuned further in Section \ref{sec:epochs}. As expected, there is a marked correlation between embedding length and evaluation score. This is due to the larger parameter space allowing more complex relationships to be learned by the model. However, this performance gain comes at a computational cost - using embedding lengths of 400 is drastically slower than 100 when measuring cross validation run time - roughly four times slower. The four fold increase will drastically slow further parameter tuning exercises. For this reason, the remainder of this project uses embedding lengths of 100, and using greater embedding lengths will be left to future work.

\begin{table}
	\caption{5 Fold cross validated scores for tuning embedding length }
	\label{tab:embedTuning}
	\centering
	\subfile{../tables/embedding_cross_val.tex}
\end{table}

\begin{table}[]
	\caption{ Number of trainable parameters for MLP and FastText models }
	\label{tab:trainParams}
	\centering
	\begin{tabular}{lllc}
		\toprule
		Model    & Domain      & Structure & Trainable Parameters \\ \midrule
		\multirow{3}{*}{MLP}      & Multi-class &           & 483,337               \\
		& Multi-label &           & 467,035               \\
		& Multi-task  &           & 467,035               \\
		\cmidrule{2-4}
		\multirow{3}{*}{FastText} & Multi-class &           & 515,537               \\
		& Multi-label &           & 452,210               \\
		& Multi-task  &           & 457,110 \\ \bottomrule              
	\end{tabular}
	
\end{table}

\subsection{Epochs} \label{sec:epochs}
When training neural networks, several passes through the training are usually required. Each pass is known as an 'epoch', and it is important to use an appropriate number of epochs. Too few and a model will not learn enough, too many and a model will over fit training data. Table \ref{tab:crossValEpoch} shows 5 fold cross validated test results by epoch length for each fastText and MLP classifier. We can see that for
the MLP multi-class classifier peak EMR and \f{Micro} (equivalent figures in single label multi-class problems) is achieved at just 50 epochs, while peak \f{Macro} occurs at 350 epochs. The MLP multi-label performs best at between 150 and 350 epochs, while the multi-task version nets best performance between 250 and 300 epochs. For the fastText multi-label classifier peak performance is reached at 250 epochs, and for multi-task peak performance is reached between 200 and 350 epochs. However, for multi-class peak performance is not achieved until 400 epochs. Note that as per table \ref{tab:trainParams} the fastText multi-class classifier has significantly more trainable parameters than other classifiers, which possibly explains why peak performance is not reached until after further training relatively. To ensure equivalent comparison, and to avoid unfairly penalising any classifier hereafter we fix the number of epochs. It is observed that with the exception of MLP multi-class and MLP multi-label all peak performance scores occur at between 200 and 400 epochs. We therefore fix epochs to the mid point of this range, at 300 epochs.




\section{Results and Analysis}
In this section we review the final results of the project after following the methodology described in Chapter \ref{chap:method} and using the hyper-parameter values found in Section \ref{sec:hypParams}. The section begins by reviewing the within class scores for each classifier set in detail, before analysis and comparison of the combined results in the following section. All results are computed using 5 fold cross-validation.

\subsection{Logistic Regression}
The results of the logistic regression classifiers are shown in Table \ref{tab:LR1} and Table \ref{tab:LR2}.
In the multi-label implementation categorical \f{1} scores of between 0.741 and 0.828 are observed. \lab{Contrariety} is the highest scoring category, and \lab{Neccessity} the lowest. Referring back to Figure \ref{fig:labelBarChart} \lab{Contrariety} is the most common category in the dataset, so the high score of the classifier in labelling utterances as such is not unusual. However, \lab{Necessity} ranks in the middle as the 5\textsuperscript{th} most common category, suggesting there may be other reasons besides sample size influencing the lower performance. Scores of 0.774, 0.775 and 0.492 are observed for \f{Macro}, \f{Micro} and EMR respectively. 

Examining the results of the multi-class logistic regression (see Figure \ref{tab:LR2}) we see scores of 0.783 and 0.645 for \f{Macro} and EMR. Notably both of these scores are higher than the multi-label implementation. This suggests that for linear classifiers such as logistic regression higher performance in multi-label problems can be achieved using multi-label to multi-class transformations such as the power set transformation used here. This is likely because in the multi-class implementation the classifier can only predict classes observed in training. 

Overall both classifiers perform remarkably well. Logistic regression is known to be a strong baseline for NLP problems, and the results observed here support this notion. In comparison to previous work in stance detection the results also compare very favourably.

\begin{table}[h!]
	\caption{5 fold cross validated categorical \f{1} scores for logisitic regression classifier}
	\label{tab:LR1}
	\centering
	\begin{tabular}{@{}lc@{}}
		\toprule
		Stance Category         & Multi-label \\ \midrule
		\lab{Agreement/Disagreement} & 0.790       \\
		\lab{Certainty}              & 0.775       \\
		\lab{Contrariety}            & \bfseries 0.828       \\
		\lab{Hypotheticality}        & 0.770       \\
		\lab{Necessity}              & 0.748       \\
		\lab{Prediction}             & 0.756       \\
		\lab{Source of knowledge}    & 0.760       \\
		\lab{Tact/Rudeness}          & 0.779       \\
		\lab{Uncertainty}            & 0.741       \\
		\lab{Volition}               & 0.791       \\ \bottomrule
	\end{tabular}

\end{table}

\begin{table}[h]
	\caption{5 fold cross validated  \f{Micro}, \f{Macro} and EMR scores for logistic regression classifier}
	\label{tab:LR2}
	\centering
	\begin{tabular}{@{}lll@{}}
		\toprule
						& Multi-class & Multi-label \\ \midrule
		\f{Macro} 		&  0.743       & 0.775       \\
		\f{Micro} 	& 0.645       &  0.775       \\
		EMR             &  0.645       & 0.492       \\ \bottomrule
	\end{tabular}
\end{table}
\subsection{Multi Layer Perceptron}
In Table \ref{tab:MLP1} and Table \ref{tab:MLP2} are the results for the MLP classifiers. Both multi-label and multi-task classifiers exhibit high variance, with categorical \f{1} scores ranging from 0.388 to 0.728 in the multi-label case and from 0.308 to 0.685 in the multi-task case. For both cases \lab{Uncertainty} is the highest scoring category, and across categories there appears to be a correlation with training set size. The multi-class implementation is seen to outperform multi-label and multi task versions in \f{Macro} and EMR with scores of 0.729 and 0.564 respectively. However, the multi-label classifier has the highest \f{Micro} score of 0.617. No MLP classifier is able to beat the logistic regression baselines, and as such further work is necessary to improve the MLP implementations.
\begin{table}[]
		\caption{Within category 5 fold cross validated \f{1} scores for MLP classifier}
		\label{tab:MLP1}
		\centering
	\begin{tabular}{@{}lll@{}}
		\toprule
		Stance Category         & Multi-label & Multi-task \\ \midrule
		\lab{Agreement/Disagreement}	& 0.388       & 0.347      \\
		\lab{Certainty}	  			& 0.518       & 0.491      \\
		\lab{Contrariety}	            & 0.612        & 0.603      \\
		\lab{Hypotheticality}			& 0.523       & 0.491      \\
		\lab{Necessity}				& 0.579       & 0.499     \\
		\lab{Prediction}				& 0.686       & 0.665      \\
		\lab{Source of knowledge}		& 0.620       & 0.624      \\
		\lab{Tact/Rudeness}			& 0.456       & 0.345      \\
		\lab{Uncertainty}				& \bfseries 0.728       & 0.685      \\
		\lab{Volition} 				& 0.501       & 0.308      \\ \bottomrule
	\end{tabular}
\end{table}
\begin{table}[]
	\centering
	\caption{5 fold cross validated \f{Macro}, \f{Micro} and EMR scores for MLP classifier}
			\label{tab:MLP2}
	\begin{tabular}{@{}llll@{}}
		\toprule
		& Multi-class & Multi-label & Multi-task \\ \midrule
		\f{Macro} & 0.742       & 0.556       & 0.509      \\
		\f{Micro} & 0.579       & 0.617       & 0.586      \\
		EMR       & 0.579       & 0.361       & 0.326      \\ \bottomrule
	\end{tabular}
\end{table}
\subsection{FastText}
Table \ref{tab:FT1} and Tabel \ref{tab:FT2} show the final results for fastText classifiers. Within the categorical \f{1} scores \lab{Uncertainty} scores highest at 0.873 and 0.870 for the multi-label and multi-task classifiers respectively. \lab{Source of Knowledge} scores lowest at 0.735 in multi-label mode and 0.731 in multi-task mode. There is a notable uniformity between the scores of the multi-label and multi-task classifiers, which could suggest that the multi-task implementation is not learning any extra information compared to the multi-label. It is likely that the multi-task approach requires further tuning or architectural changes. Similarly the multi-label and multi-task classifiers have identical scores when measuring \f{Macro} and \f{Micro}, 0.799 and 0.801 respectively. These are the highest achieved by any classifier in the project. The low difference between \f{Macro} and \f{Micro} scores shows that the classifiers are performing almost equally between frequent and rarer categories. This suggests that both approaches are good ones, and are approaching the optimal solution to the problem. However in EMR score both implementations lag behind the multi-class fastText classifier , which achieves 0.650. Again, this is the highest EMR achieved by any classifier in the project.
\begin{table}[]
	\centering
			\caption{Within category 5 fold cross validated \f{1} scores for fastText classifier}
			\label{tab:FT1}
	\begin{tabular}{@{}lll@{}}
		\toprule
		Stance Category         & Multi-label & Multi-task \\ \midrule
		\lab{Agreement/Disagreement} & 0.819       & 0.819      \\
		\lab{Certainty}              & 0.848       & 0.852      \\
		\lab{Contrariety}            & 0.798         & 0.797      \\
		\lab{Hypotheticality}        & 0.755       & 0.760      \\
		\lab{Necessity}              & 0.750        & 0.749      \\
		\lab{Prediction}             & 0.838       & 0.846      \\
		\lab{Source of knowledge}    & 0.735       & 0.731      \\
		\lab{Tact/Rudeness}          & 0.801       & 0.801      \\
		\lab{Uncertainty}            & 0.873       & 0.870      \\
		\lab{Volition}               & 0.777       & 0.765      \\ \bottomrule
	\end{tabular}
\end{table}


\begin{table}[]
	\centering
		\caption{5 fold cross validated \f{Macro}, \f{Micro} and EMR scores for fastText classifier}
		\label{tab:FT2}
	\begin{tabular}{@{}llll@{}}
		\toprule
		& Multi-class & Multi-label & Multi-task \\ \midrule
		\f{Macro} & 0.766       & 0.799       & 0.799      \\
		\f{Micro} & 0.650       & 0.801        & 0.801      \\
		EMR             & 0.650       & 0.549       & 0.548      \\ \bottomrule
	\end{tabular}
\end{table}
\section{Analysis}
\begin{table}
	\caption{Table of final results for all classifiers}
	\label{tab:FinalResults}
	\centering
	\subfile{../tables/all_results.tex}
\end{table}
Table \ref{tab:FinalResults} compares \f{Macro}, \f{Micro} and EMR for all implementations of the three classifier types. Overall, the results of each classifier compare well to previous work in the literature. Between architectures the multi-layer perceptron shows the most need for improvement, producing the lowest scores of the three architectures.  Logistic regression provides an excellent baseline, and indeed the multi-class implementation scores the second highest exact match ratio (EMR) while the multi-label implementation scores the second highest \f{Macro} and \f{Micro} scores. However, the fastText classifiers show the most promise and is the best architecture across all measures. Notably, the fastText classifiers show very little variance between stance categories, showing how using representation learning (word embeddings) instead of feature engineering helps to better capture semantic similarity between instances and somewhat uncouple the correlation between category frequency in the training set and classifier categorical \f{1} score.



\subsection{Anchors}
TBC
\begin{table}[h]
	\caption{5 fold cross validated \f{1} and EMR scores for tuning epoch hyper-parameter}
	\label{tab:crossValEpoch}
	\centering
	\begin{subfigure}{0.5\linewidth}
		\centering
		\subfile{../tables/epochs.tex}	
		\subcaption{FastText}	
	\end{subfigure}\hspace*{1em}\begin{subfigure}{0.5\linewidth}
		\centering
		\subfile{../tables/epochs2.tex}	
		\subcaption{Multi-layer perceptron}	
	\end{subfigure}
	
\end{table}


\end{document}